\section{Barrelfish Overview%
  \label{barrelfish-overview}%
}

Barrelfish currently runs on:
%
\begin{quote}
%
\begin{itemize}

\item x86 CPUs in either IA-32 or AMD64 mode. The following are known to work:
%
\begin{itemize}

\item Intel Xeon Clovertown, Gainestown, Beckton (X5355, E5520, X7560, L5520,
L7555)

\item AMD Opteron Santa Rosa, Barcelona, Shanghai, Istanbul, Magny Cours
(2220, 8350, 8374, 8380, 8431, 6174)

\item QEMU simulator

\end{itemize}

\item Intel Single-Chip Cloud Computer (SCC), both Copper Ridge and Rocky Lake
are known to work.

\item ARM Platform
- ARMv7 and ARMv5 architectures
- GEM5 simulator
- Pandaboard System On Chip platform

\end{itemize}

\end{quote}

This README file documents instructions for x86 architecture.  For other
architectures, please refer to the corresponding technote documentations.
You can either generate the latest documentation from this source-code
(instructions at end of this file), or visit the \href{http://www.barrelfish.org/}{Barrelfish} website
to download them.


\section{Supported PC hardware%
  \label{supported-pc-hardware}%
}

Barrelfish supports following PC hardware :
%
\begin{quote}
%
\begin{itemize}

\item x86 CPUs in either IA-32 or AMD64 mode. The following are known to work:
%
\begin{itemize}

\item Intel Xeon Clovertown, Gainestown, Beckton (X5355, E5520, X7560, L5520,
L7555)

\item AMD Opteron Santa Rosa, Barcelona, Shanghai, Istanbul, Magny Cours
(2220, 8350, 8374, 8380, 8431, 6174)

\end{itemize}

\end{itemize}

\end{quote}

The biggest compatibility problems are likely to be in the PCI/ACPI code. We
usually discover new quirks (or missing functionality in the ACPI glue code)
on each new machine we test. The following systems are known to work:
%
\begin{quote}
%
\begin{itemize}

\item Intel x5000XVN

\item Tyan n6650W and S4985

\item Supermicro H8QM3-2

\item Dell PowerEdge R610 and R905

\item Sun Fire X2270 and X4440

\item Intel/Quanta QSSC-S4R

\item Lenovo X200 and X301 laptops

\item ASUS Eee PC 1015PEM netbooks

\end{itemize}

\end{quote}

The e1000n driver should work with most recent Intel gigabit ethernet
controllers (see the list in devices/e1000.dev). We've mostly used the
82572EI (PCI device ID 0x1082).

You should also be able to boot Barrelfish on a recent version of QEMU (0.14);
note that the e1000 device emulated by QEMU is not supported by our driver.


\section{Required Tools%
  \label{required-tools}%
}

The following are required to build Barrelfish and its tools:
%
\begin{quote}
%
\begin{itemize}

\item GCC 4.x
%
\begin{itemize}

\item 4.4.5, and 4.5.2 are known to work

\item cross-compiling between i386 and x86\_64 works (requires libc6-dev-i386
to build 32 bit on 64 bit machine)

\item for the ARM port, we recommend the EABI tools available from \href{http://www.codesourcery.com/sgpp/lite/arm}{CodeSourcery}.

\end{itemize}

\item GNU binutils (2.19 is known to work)

\item GNU make

\item GHC v7.4 and Parsec 3.1
- older versions of the tree supported v6.10 or v6.12.2 with Parsec 2.1
- GHC v6.12.1 has a known bug and is unable to build our tools
- earlier versions of GHC are unsupported

\end{itemize}

\end{quote}

Our build system may not be very portable; if in doubt, try building on a
recent Debian or Ubuntu system, as these are what we use.


\section{Building%
  \label{building}%
}
\newcounter{listcnt0}
\begin{list}{\arabic{listcnt0}.}
{
\usecounter{listcnt0}
\setlength{\rightmargin}{\leftmargin}
}

\item Assuming you have already unpacked the sources, create a build directory
%
\begin{quote}{\ttfamily \raggedright \noindent
\$~mkdir~build~\&\&~cd~build
}
\end{quote}
\end{list}

1. Run \texttt{hake.sh}, giving it the path to the source directory and target
architecture(s)
%
\begin{quote}{\ttfamily \raggedright \noindent
\$~../hake/hake.sh~-s~../~-a~x86\_64
}
\end{quote}

This will configure the build directory and use GHC to compile and then run
hake, a tool used to generate the \texttt{Makefile}.

3. Optionally, edit the configuration parameters in \texttt{hake/Config.hs} and
run \texttt{make rehake} to apply them.
\setcounter{listcnt0}{0}
\begin{list}{\arabic{listcnt0}.}
{
\usecounter{listcnt0}
\addtocounter{listcnt0}{3}
\setlength{\rightmargin}{\leftmargin}
}

\item Run make, and wait
%
\begin{quote}{\ttfamily \raggedright \noindent
\$~make
}
\end{quote}

\item If everything worked, you should now be able to run Barrelfish inside QEMU
%
\begin{quote}{\ttfamily \raggedright \noindent
\$~make~sim
}
\end{quote}
\end{list}


\section{Installing and Booting%
  \label{installing-and-booting}%
}

Barrelfish requires a Multiboot-compliant bootloader that is capable of loading
an ELF64 image. At the time of writing, this doesn't include the default GRUB.
Your options are either:
%
\begin{quote}
%
\begin{itemize}

\item use the pre-loader ``elver'' that can be found in the tools directory

\item patch GRUB to support a 64-bit kernel image, using this \href{http://savannah.gnu.org/bugs/?17963}{patch}.

\end{itemize}

\end{quote}

``Installing'' Barrelfish currently consists of copying the ELF files for the CPU
driver and user programs to a location that the target machine can boot from,
and writing a suitable menu.lst file that instructs the bootloader (GRUB) which
programs to load and the arguments to pass them.

If you specify an appropriate INSTALL\_PREFIX, \texttt{make install} will copy the
binaries to the right place for you, eg
%
\begin{quote}{\ttfamily \raggedright \noindent
\$~make~install~INSTALL\_PREFIX=/tftpboot/barrelfish
}
\end{quote}

We usually boot Barrelfish via PXE/TFTP, although loading from a local disk
also works. Instructions for setting up GRUB to do this are beyond the scope of
this document. Assuming you have such a setup, here is a sample menu.lst file
for a basic diskless boot that doesn't do anything useful beyond probing the
PCI buses and starting a basic shell
%
\begin{quote}{\ttfamily \raggedright \noindent
title~~~Barrelfish\\
root~~~~(nd)\\
kernel~/barrelfish/x86\_64/sbin/elver\\
module~/barrelfish/x86\_64/sbin/cpu\\
module~/barrelfish/x86\_64/sbin/init\\
module~/barrelfish/x86\_64/sbin/mem\_serv\\
module~/barrelfish/x86\_64/sbin/monitor\\
module~/barrelfish/x86\_64/sbin/ramfsd~boot\\
module~/barrelfish/x86\_64/sbin/skb~boot\\
modulenounzip~/barrelfish/skb\_ramfs.cpio.gz~nospawn\\
module~/barrelfish/x86\_64/sbin/acpi~boot\\
module~/barrelfish/x86\_64/sbin/pci~boot\\
module~/barrelfish/x86\_64/sbin/spawnd~boot\\
module~/barrelfish/x86\_64/sbin/serial\\
module~/barrelfish/x86\_64/sbin/fish
}
\end{quote}

There are many other programs you can load (take a look around the usr tree for
examples). To start a program on a core other than the BSP core, pass
\texttt{core=N} as its first argument.

If things work, you should see output on both the VGA console and COM1.


