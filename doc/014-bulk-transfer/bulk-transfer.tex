%%%%%%%%%%%%%%%%%%%%%%%%%%%%%%%%%%%%%%%%%%%%%%%%%%%%%%%%%%%%%%%%%%%%%%%%%%
% Copyright (c) 2011, ETH Zurich.
% All rights reserved.
%
% This file is distributed under the terms in the attached LICENSE file.
% If you do not find this file, copies can be found by writing to:
% ETH Zurich D-INFK, Haldeneggsteig 4, CH-8092 Zurich. Attn: Systems Group.
%%%%%%%%%%%%%%%%%%%%%%%%%%%%%%%%%%%%%%%%%%%%%%%%%%%%%%%%%%%%%%%%%%%%%%%%%%

\documentclass[a4paper,twoside]{report} % for a report (default)

\usepackage{bftn} % You need this

\title{Bulk Transfer}   % title of report
\author{Pravin Shinde} % author
\tnnumber{014}  % give the number of the tech report
\tnkey{Bulk Transfer: The mechanism to transfer large data} 
% Short title, will appear in footer

% \date{Month Year} % Not needed - will be taken from version history

\begin{document}
\maketitle

%
% Include version history first
%
\begin{versionhistory}
\vhEntry{1.0}{11.08.2011}{TR}{Initial version}
\end{versionhistory}

% \intro{Abstract}      % Insert abstract here
% \intro{Acknowledgements}  % Uncomment (if needed) for acknowledgements
% \tableofcontents        % Uncomment (if needed) for final draft
% \listoffigures        % Uncomment (if needed) for final draft
% \listoftables         % Uncomment (if needed) for final draft

\chapter{Introduction}
This technical note is aimed to document the design of ideal cross domain bulk
transfer mechanism that we wish to have in Barrelfish.


\section{Existing implementation in Barrelfish}
There are two bulk transfer mechanisms that exist in the current(as of August
2011) implementation of Barrelfish.
\texttt{$lib/barrelfish/bulk\_transfer.c$}
and \textit{pbufs} used in the network stack.  We will discuss them briefly in
following two sub-sections.

\subsection{$bulk\_transfer.c/h$}
This is the formal bulk transfer method supported by Barrelfish.  In brief, this 
facility works by sharing a continuous piece of memory (in form of capability) 
between two domains.  These domains then map this physical memory into their 
virtual address-space.  The virtual address where this shared memory is loaded
can be different.  So, participating processes can't exchange the virtual 
addresses directly.  Bulk transfer mechanism works by dividing the entire 
memory area into the blocks of fixed size.  The domain which initiated the 
bulk transfer is responsible for managing these blocks.  This responsibility
includes following.
\begin{enumerate}
  \item Allocation and deallocation of the blocks.
  \item Maintaining the information about free and used blocks.
\end{enumerate}
So, this bulk transfer facility works in master-slave setup.  The master
allocates the blocks and passes the index of the block to slave.  
Once the block is passed to the slave, master should not touch it.  
The slave can locate the block by adding the \texttt{(index*block-size)} to the
virtual address pointing to the beginning of the shared memory area.
Now slave can read/modify the contents in the memory area of this block.  Once
the slave is done with accessing the block, it can pass the block-index
back to the master.  Master can either access the block or release it,
so that it will be added to the list of free blocks.

\subsubsection{Limitations}
\paragraph{Security}
This transfer mechanism can only be used between domains which are
co-operative and willing to follow the protocol set for using the bulk
transfer mechanism.  Malicious domain can corrupt the data in shared memory
or can refuse to return the blocks once passed to it.

\paragraph{More than two domains}
The design of this bulk transport does not stop you from using it with
more than two domains as long as the protocol is followed.  But current
implementation does not track which domain is holding the buffer. So,
one can use the current implementation with multiple domains as long
as the domains are co-operative and applications are willing to deal
with added complexity of tracking which domains hold which buffers.

\paragraph{Use}
Due to the relatively simple nature of the implementation, its use is
limited only to few places. And hence it is not throughly tested in
various scenarios.  

\subsection{pbufs}
The network stack uses its own custom bulk transport mechanisms.  Lets
call them pbufs.  These pbufs work in similar way to above mechanism,
but are more flexible.  The application shares some piece of physical memory
with network driver which is used as shared memory.  Then application
creates a list of pbuf structures, each one of them holding an offset
within shared memory, length, \textit{pbuf-id} and the shared memory id to
which they belong.  The key difference here is that the these pbufs 
may not hold consecutive memory locations even when \textit{pbuf-id}'s are
consecutive.  In contrast with Barrelfish bulk-transfer where buffer-id
value is enough to find the location of buffer within shared memory,
pbuf needs to store the offset separately in another list.  So,
pbufs just provide another layer of indirection to allow more
flexible use of memory.


The way this mechanism currently works is that, application creates
pool of initial pbufs and registers them with network driver.  Both,
application and network driver maintain the list of pbufs in their
private memory.  This list is kept in synchronization by sending
explicit messages.  Now each pbuf can in principle point to any
buffer of any size, located anywhere in the shared memory.


This flexibility is used by application when it receives a data from
the network driver.  Application creates a new pbuf structure with
same pbuf-id but pointing at new buffer location and send it back to
the driver as new free pbuf to use.  And the location of previous
buffer is used for processing the data.  This way, application can
return the pbufs back to driver ASAP without getting affected by how
long does the data processing takes.  When application is done with
processing the data in that buffer, it releases that buffer.  This
released buffer is then used to create new pbuf which will be
registered again with driver in future. 

\subsubsection{Limitations}

\paragraph{Needs more memory}
The shared memory needs to be bigger than the memory shared with
network driver in form of registered pbufs.  This is because, at any
given point in time, some pbufs will be in application data processing
phase and hence can-not be used by driver to receive the new data.

\paragraph{Data corruption}
This bulk-transfer mechanism assumes co-operative domains.  Both
domains have read/write access to the shared memory at all the time.
This means that if they do not follow the protocol correctly, they may
end up writing at same physical location, leading to data corruption.

\paragraph{Complicated memory reclamation}
The current implementation of pbufs assumes that application can have
multiple threads and all of them can access the data in pbufs which
are delivered to the application.  This complicates the problem of
detecting when exactly all threads are done with accessing the
particular pbuf.  The current implementation uses a reference count
mechanism for this detection.  Even though this memory reclamation is
functional, it is complicated and one can easily get it wrong, leading
to memory leaks.

\paragraph{Supporting more than two domains}
In theory, this design can be used with more than two domains which
are co-operating with each other, but implementation is not designed
with such a case in mind.  The problematic issues will be tracking
which memory buffer is with which domain, and when the memory buffer
can be reused.

\section{Questions to be asked about bulk transfer mechanisms}
\begin{enumerate}
  \item Can it lead to deadlock if one domain is only sending without
  checking for arriving messages?
  \item Can consumer/producer put a back-pressure on other side?
\end{enumerate}

\section{Desired properties from bulk transfer mechanisms}

\end{document}
